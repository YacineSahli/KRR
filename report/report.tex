%%
%% This is file `sample-acmsmall-conf.tex',
%% generated with the docstrip utility.
%%
%% The original source files were:
%%
%% samples.dtx  (with options: `acmsmall-conf')
%%
%% IMPORTANT NOTICE:
%%
%% For the copyright see the source file.
%%
%% Any modified versions of this file must be renamed
%% with new filenames distinct from sample-acmsmall-conf.tex.
%%
%% For distribution of the original source see the terms
%% for copying and modification in the file samples.dtx.
%%
%% This generated file may be distributed as long as the
%% original source files, as listed above, are part of the
%% same distribution. (The sources need not necessarily be
%% in the same archive or directory.)
%%
%% The first command in your LaTeX source must be the \documentclass command.
\documentclass[acmsmall]{acmart}

\usepackage{listings}
\usepackage{amsmath}
\usepackage{float}

%%
%% \BibTeX command to typeset BibTeX logo in the docs
\AtBeginDocument{%
  \providecommand\BibTeX{{%
    \normalfont B\kern-0.5em{\scshape i\kern-0.25em b}\kern-0.8em\TeX}}}

%% Rights management information.  This information is sent to you
%% when you complete the rights form.  These commands have SAMPLE
%% values in them; it is your responsibility as an author to replace
%% the commands and values with those provided to you when you
%% complete the rights form.
\setcopyright{acmcopyright}
\copyrightyear{2020}
\acmYear{2020}

%% These commands are for a PROCEEDINGS abstract or paper.
\acmConference[Galway '20]{Galway '20: ACM International Conference on Information and Knowledge Management}{October 19--23, 2020}{Galway, Ireland}
\acmBooktitle{Galway '20: ACM International Conference on Information and Knowledge Management,
  October 19--23, 2020, Galway, Ireland}


%%
%% Submission ID.
%% Use this when submitting an article to a sponsored event. You'll
%% receive a unique submission ID from the organizers
%% of the event, and this ID should be used as the parameter to this command.
%%\acmSubmissionID{123-A56-BU3}

%%
%% The majority of ACM publications use numbered citations and
%% references.  The command \citestyle{authoryear} switches to the
%% "author year" style.
%%
%% If you are preparing content for an event
%% sponsored by ACM SIGGRAPH, you must use the "author year" style of
%% citations and references.
%% Uncommenting
%% the next command will enable that style.
%%\citestyle{acmauthoryear}

%%
%% end of the preamble, start of the body of the document source.
\begin{document}

%%
%% The "title" command has an optional parameter,
%% allowing the author to define a "short title" to be used in page headers.
\title{ASP for Consistent Query Answering}

%%
%% The "author" command and its associated commands are used to define
%% the authors and their affiliations.
%% Of note is the shared affiliation of the first two authors, and the
%% "authornote" and "authornotemark" commands
%% used to denote shared contribution to the research.

\author{Yacine Sahli}
\affiliation{%
  \institution{University of Mons}
  \city{Mons}
  \country{belgium}
}

\author{Joachim Sneessens}
\affiliation{%
  \institution{University of Mons}
  \city{Mons}
  \country{belgium}
}

\author{Maxime Daniels}
\affiliation{%
  \institution{University of Mons}
  \city{Mons}
  \country{belgium}
}

%%
%% By default, the full list of authors will be used in the page
%% headers. Often, this list is too long, and will overlap
%% other information printed in the page headers. This command allows
%% the author to define a more concise list
%% of authors' names for this purpose.
\renewcommand{\shortauthors}{Sahli, Sneesens and Daniels.}

%%
%% The abstract is a short summary of the work to be presented in the
%% article.
\begin{abstract}
	Consistent query answering for inconsistent databases is a running problem. We realized an ASP implementation of the consistent query answering problem and ran some experiments comparing a generate and test method against a first-order rewriting.
\end{abstract}

%%
%% The code below is generated by the tool at http://dl.acm.org/ccs.cfm.
%% Please copy and paste the code instead of the example below.
%%
\begin{CCSXML}
\ccsdesc[500]{Computer systems organization~Embedded systems}
\ccsdesc[300]{Computer systems organization~Redundancy}

\begin{CCSXML}
<ccs2012>
   <concept>
       <concept_id>10002951.10002952.10002953</concept_id>
       <concept_desc>Information systems~Database design and models</concept_desc>
       <concept_significance>500</concept_significance>
       </concept>
   <concept>
       <concept_id>10002951.10002952.10003190.10003192</concept_id>
       <concept_desc>Information systems~Database query processing</concept_desc>
       <concept_significance>500</concept_significance>
       </concept>
 </ccs2012>
\end{CCSXML}

\ccsdesc[500]{Information systems~Database design and models}
\ccsdesc[500]{Information systems~Database query processing}
%%
%% Keywords. The author(s) should pick words that accurately describe
%% the work being presented. Separate the keywords with commas.
\keywords{Answer Set Programming, Consistent Query Answering}

%% A "teaser" image appears between the author and affiliation
%% information and the body of the document, and typically spans the
%% page.
\begin{teaserfigure}
  \includegraphics[width=\textwidth]{sampleteaser}
  \caption{}
  \Description{}
  \label{fig:teaser}
\end{teaserfigure}

%%
%% This command processes the author and affiliation and title
%% information and builds the first part of the formatted document.
\maketitle

\newpage

\section{Related Work}
\emph{Consistent query answering} (CQA) started in 1999 with a seminal paper by Arenas, Bertossi and Chomicki~\cite{DBLP:conf/pods/ArenasBC99}.
Two decades of research in CQA have recently been surveyed in~\cite{DBLP:conf/pods/Bertossi19,DBLP:journals/sigmod/Wijsen19}. 

The suitability of Answer Set Programming (ASP) and stable model semantics for CQA has been recognized since the early days in theoretical research~\cite{DBLP:journals/tplp/ArenasBC03,DBLP:journals/tkde/GrecoGZ03}.
In~\cite{DBLP:journals/dke/MarileoB10}, the authors present a prototype system for CQA that is theoretically founded in ASP.

In CQA, the existence of consistent first-order rewritings, for different classes of queries and integrity constraints, is a recurrent research problem.
For self-join-free conjunctive queries and primary keys, the problem has been studied in depth since~2005~\cite{DBLP:conf/icdt/FuxmanM05,DBLP:journals/jcss/FuxmanM07}, and was eventually solved in~2015~\cite{DBLP:conf/pods/KoutrisW15,DBLP:journals/tods/KoutrisW17}.
Experiments of CQA with respect to primary keys have been conducted on several prototype systems, including ConQuer~\cite{DBLP:conf/sigmod/FuxmanFM05}, EQUIP~\cite{DBLP:journals/pvldb/KolaitisPT13}, and CAvSAT~\cite{DBLP:conf/sat/DixitK19}.
The latter study, in particular, has revealed that discrepancies may exists between the theoretical computational complexity of $\cqa{q}$ and observed empirical performances. 
In particular, it was observed that a generic SAT-based approach to $\cqa{q}$ may outperform solutions that use first-order rewriting.
The main goal of the current paper is to investigate whether such observations also hold within the system of clingo ASP~\cite{DBLP:conf/lpnmr/GebserKKS11,DBLP:journals/corr/GebserKKS14}.



\section{Introduction}
The aim of this article is to present a fair comparison between two methods for solving the problem of CERTAINTY(q). Considering an inconsistent database, a repair is a maximal set of tuples from this database that respects his constraints. The CERTAINTY(q) problem consists in answering the question of knowing if it exists a repair that falsifies the query. Depending on the query, the CERTAINTY(q) problem can be either in first order complexity class, or in NP or co-NP. For the queries that are in first order, we want to compare the efficiencies of the generate-and-test method and of the first order rewriting method. 



\section{Chosen queries}

$$q_1(z) := \exists x, y, v, w (R_1(\underline{x},y,z) \wedge R_2(\underline{y}, v, w)) $$

$$q_2(z,w) :=  \exists x, y, v (R_1(\underline{x},y,z) \wedge R_2(\underline{y}, v, w)) $$

$$q_3(z) :=  \exists x, y, v, u, d (R_1(\underline{x},y,z) \wedge R_3(\underline{y}, v) \wedge R_2(\underline{v}, u,d)) $$

$$q_4(z,d) :=  \exists x, y, v, u (R_1(\underline{x},y,z) \wedge R_3(\underline{y}, v)\wedge R_2(\underline{v}, u, d)) $$

$$q_5(z) := \exists x, y, v, w (R_1(\underline{x},y,z) \wedge R_4(\underline{y, v},w)) $$

$$q_6(z) := \exists x, y, x', w, d (R_1(\underline{x},y,z) \wedge R_2(\underline{x'},y,w))
\wedge R_5(\underline{x},y,d) $$

$$q_7(z) := \exists x, y, w, d (R_1(\underline{x},y,z)
\wedge R_2(\underline{y},x,w) \wedge R_5(\underline{x},y,d)) $$




\section{First query}

\begin{lstlisting}
certainty(Z):-r1(X,Y,Z),not p0(X,Z),not p1(x).
p0(X,Z):-r1(X,Y,Z1),r1(X,_,Z),not Z=Z1.
p1(X):-r1(X,Y,Z1),not p2(Y).
p2(Y):-r2(Y,V,W).

#show certainty/1.
\end{lstlisting}

\section{Second query}

\begin{lstlisting}
certainty(W,Z):-r1(X,Y,Z),not p0(Z,X),not q0(W,X), r2(P,Q,W).
p0(Z,X):-r1(X,Y,Z1),not Z1=Z,r1(X,_,Z).
q0(W,X):-r1(X,Y,Z1),not q1(W,Y), r2(P,Q,W).
q1(W,Y):-r2(Y,V,W),not q2(W,Y).
q2(W,Y):-r2(Y,V,W1),not W1=W,r2(Y,_,W).

#show certainty/2.
\end{lstlisting}

\section{Fourth query}

Generate-and-test method. (Does not work yet).

\begin{lstlisting}
1 {rr1(X,Y,Z) : r1(X,Y,Z)} 1 :- r1(X,_,_).

1 {rr4(X,Y,Z) : r4(X,Y,Z)} 1 :- r4(X,Y,_).

:- rr1(X,Y,Z), rr4(Y,V,W). 
\end{lstlisting}


FO rewriting

\begin{lstlisting}
p(X,Z) :- r1(X,Y,Z2), Z2!=Z, r1(X,Y2,Z).
t(X) :- r1(X,Y,Z), not q(Y).
q(Y) :- r4(Y,V,W).
answer(Z) :- r1(X,Y,Z), not p(X,Z), not t(X).

#show answer/1.
\end{lstlisting}

\section{Seventh query}


FO rewriting

\begin{lstlisting}
certainty(Z) :- not d1(Z,Y), r2(Y,X,W), r1(X,Y,Z).
d1(Z,Y) :- not d2(Z,Y,X,W), r2(Y,X,W), r1(X,Y,Z).
d2(Z,Y,X,W) :- not d3(Z,Y,X,W), r2(Y,X,W), r1(X,Y,Z).
d3(Z,Y,X,W) :- r2(Y,X,W), not d4(Z,Y,X,W,P,Q), r1(X,P,Q), r1(X,Y,Z).
d4(Z,Y,X,W,P,Q) :- r1(X,P,Q), P=Y, r2(Y,X,W), Q=Z, d5(Z,Y,X,W).
d5(Z,Y,X,W) :- r5(X,Y,D), not d6(Z,Y,X,W), r2(Y,X,W), r1(X,Y,Z).
d6(Z,Y,X,W) :- not d7(Z,Y,X,W,P,D), r2(Y,X,W), r5(X,P,D), r1(X,Y,Z).
d7(Z,Y,X,W,P,D) :- r2(Y,X,W), r5(X,Z_5_0,D), r1(X,Y,Z), P=Y.

#show certainty/1.
\end{lstlisting}




\section{Results}

For each graph, the blue line corresponds to the time taken by the
generate-and-test method and the orange line to the time taken by the FO
method. When a result is not in the graph, that means that the execution of the
program was interrupted for insufficient memory. For example, the times for the
generate-and-test for the yes-instance for $q1$ for the databases size greater
than 1 million are absent.

All the databases used as 20 \% of inconsistency.

\begin{figure}[H]
\includegraphics[width=0.6\textwidth]{time_q1_yesinstance.png}
\centering
\end{figure}

\begin{figure}[H]
\includegraphics[width=0.6\textwidth]{time_q1_noinstance.png}
\centering
\end{figure}

\begin{figure}[H]
\includegraphics[width=0.6\textwidth]{time_q2_yesinstance.png}
\centering
\end{figure}

\begin{figure}[H]
\includegraphics[width=0.6\textwidth]{time_q2_noinstance.png}
\centering
\end{figure}

\begin{figure}[H]
\includegraphics[width=0.6\textwidth]{time_q3_yesinstance.png}
\centering
\end{figure}

\begin{figure}[H]
\includegraphics[width=0.6\textwidth]{time_q3_noinstance.png}
\centering
\end{figure}

We see that the fo rewriting leads to better results, in terms of cpu time, that the generate-and-test method.

\section{Conclusion}

We rewrited 3 first-order rewritable queries in ASP and showed that the fo rewriting are more efficient than a generate and test method. The first order rewriting consumes a lot less memory than the generate and test which cannot run on 8GB of RAM on a database containing 3 millions or more entries.


\appendix



\end{document}
\endinput

%%
%% End of file `sample-acmsmall-conf.tex'.
